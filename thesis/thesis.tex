% Paper, encoding and fonts settings
\documentclass[12pt, twoside, a4paper, openright]{report}
\usepackage[top=2.5cm, bottom=2.5cm, inner=3cm, outer=2.5cm]{geometry}
\usepackage[T1]{polski}
\usepackage{helvet}
\renewcommand{\familydefault}{\sfdefault}
\usepackage[utf8]{inputenc}
\usepackage{indentfirst}
\setlength{\parindent}{10mm}

% Document structure and layout
\usepackage{pdfpages}
\usepackage{emptypage}
\usepackage[toc, page]{appendix}
\usepackage{fancyhdr}
\usepackage{setspace}
\usepackage{titlesec}
\linespread{1.3}

% Proper decimal separation
\usepackage{icomma}
% Math, physics and numbers
\usepackage{amsmath}
\usepackage{mathtools}
\usepackage{amssymb}
\usepackage{siunitx}
\sisetup{locale = FR}

% Graphics and plotting
\usepackage{graphicx}
\usepackage[all]{nowidow}
\usepackage{rotating}
\usepackage{makecell}
\renewcommand\theadalign{bc}
\graphicspath{ {graphics/} }

\usepackage{diagbox}
\usepackage[justification=centering]{caption}
\usepackage{subcaption}
\usepackage{array}

\usepackage[hidelinks]{hyperref}
\usepackage[shortlabels]{enumitem}

\usepackage{minted}
\renewcommand\listoflistingscaption{Spis listingów}
\usepackage{booktabs}
\usepackage{multirow}
\titleformat{\chapter}[hang]{\huge\bfseries}{\chaptertitlename\ \thechapter.}{0.3em}{}
\titlespacing*{\chapter}{0pt}{0pt}{40pt}

\renewcommand{\appendixtocname}{Dodatki}
\renewcommand{\appendixpagename}{Dodatki}

\newcolumntype{C}[1]{>{\centering\arraybackslash}p{#1}}

% Define header style
\fancyhead{}
\setlength{\headheight}{16pt}
\fancyhead[RO]{\nouppercase{\rightmark}}
\fancyhead[LE]{\nouppercase{\leftmark}}
\pagestyle{fancy}


\begin{document}

\newpage
\thispagestyle{empty}
\newgeometry{top=2.5cm, bottom=2.5cm, left=3cm, right=2.5cm}
\begin{onehalfspacing}
\begin{center}
	% TODO: vspaces are coarsely approximated, correct them
	\includegraphics[scale=0.75]{polsl_logo_bw_pl}
	\vspace{0.8cm}
	
	\fontsize{18}{18} \selectfont
	\textbf{\textsc{Politechnika Śląska \linebreak
	Wydział Automatyki, Elektroniki i~Informatyki \linebreak
	Kierunek Automatyka i~Robotyka}}
	\vspace{1.3cm}
	
	\fontsize{16}{16} \selectfont
	Projekt inżynierski
	\vspace{1.7cm}
	
	\fontsize{14}{14} \selectfont
	Sprzętowa implementacja regulatora MPC
	\vspace{5cm}
	
	\begin{flushleft}
	Autor: Szymon Zosgórnik \linebreak
	Kierujący pracą: dr hab. inż., prof. PŚ Jarosław Śmieja \linebreak
	\end{flushleft}
	
	\vfill
	\fontsize{12}{12} \selectfont
	Gliwice, styczeń 2020
\end{center}
\end{onehalfspacing}
\restoregeometry


\cleardoublepage

\pagenumbering{gobble}

\cleardoublepage

\includepdf[pages=-]{declaration.pdf}

\cleardoublepage

% Introductory pages

\begin{abstract}
    Tematem pracy jest sprzętowa implementacja algorytmu sterowania predykcyjnego (MPC).
    Regulacja MPC jest to proces złożony obliczeniowo. W związku z tym istnieje ryzyko
    braku odpowiedzi mikrokontrolera w wyznaczonym czasie. Postanowiono przyjąć pewne
    obostrzenia w celu sprawdzenia wpływu parametrów regulatora na jakość odpowiedzi
    układu. Założono znajomość dyskretnego modelu obiektu. Ponadto przyjęto jego
    liniowy charakter. Zaimplementowano stosowny algorytm regulacji predykcyjnej 
    przy użyciu języka programowania C++ oraz platformy STM. Przeprowadzono testy
    Hardware in the loop (HIL) z wykorzystaniem skryptu napisanego w języku
    programowania Python mające na celu zweryfikowanie poprawności zaimplementowanego
    algorytmu. Zebrane obserwacje potwierdziły teorię regulacji predykcyjnej.
    Efekty pracy mogą stanowić punkt wyjścia dla rozwoju opracowanego przykładu
    o bardziej zaawansowane metody, takie jak identyfikacja sterowanego obiektu.
\end{abstract}
 
\tableofcontents

\newpage
\pagenumbering{arabic}

% Chapters
\chapter{Wstęp}
\section{Motywacja projektu}
Lorem ipsum.

\section{Cel pracy}
Lorem ipsum.
      

\chapter{Idea regulatora MPC} \label{ch:idea}
\section{Wstęp}
Model predictive control (MPC) jest to zaawansowana metoda sterowania, która
polega na takim dobraniu sterowania, aby spełniało ono szereg ograniczeń. Od lat 80. XX wieku
algorytm ten wykorzystywany jest przemyśle procesowym w~zakładach chemicznych i~rafineriach 
ropy naftowej. W ostatnich latach MPC znalazło zastosowanie także w elektrowniach i~elektronice
mocy. Sterowanie predykcyjne wykorzystuje dynamiczny model obiektu, najczęściej jest to empiryczny
model pozyskany za pomocą indetyfikacji systemów. Główną zaletą MPC jest optymalizacja obecnego
przedziału czasowego, biorąc pod uwagę przyszłe stany obiektu. Jest to osiągnięte poprzez
optymalizację skończonego horyzonu czasowego, ale z~wykorzystaniem jedynie sterowania wyliczonego
dla obecnej chwili czasu. Proces ten jest powtarzany z każdą iteracją algorytmu rozwiązującego
układ równań różniczkowych opisujących dany układ. Taki schemat regulacji powoduje, że istnieje
możliwość przewidzenia przyszłych zdarzeń (występujących zgodnie z podanym modelem wartości zadanej)
i podjęcia odpowiednich działań regulujących pracę układem we wcześniejszych chwilach. Sterowanie
predykcyjne jest zazwyczaj zaimplementowane jako dyskretny regulator, lecz obecnie prowadzone są
badania mające na celu uzyskanie szybszej odpowiedzi przy użyciu specjalnie do tego przygotowanych
układów analogowych.

\section{Sposób działania} \label{sec:howitworks}
Zasada pracy regulatora MPC polega na minimalizacji różnic między wartościami predykowanymi:
$y_{k+i|k}$ w chwili obecnej $k$ na przyszłą $k+i$, a wartościami zadanymi dla tych wyjść $r(i)$.
Przez minimalizację tychże różnic rozumiana jest minimalizacja określonego kryterium jakości $J$. W
następnej chwili czasu $(k+1)$ następuje kolejny pomiar sygnału na wyjściu obiektu, a cała procedura
powtarzana jest z takim samym horyzontem predykcji $N_{p}$. W tym celu stosowana jest
więc zasada sterowania repetycyjnego bazującego na przesuwnym horyzoncie czasu. W algorytmie regulacji MPC
obecny jest także tzw. horyzont sterowania $N_{c}$ (gdzie $N_{c} \leqslant N_{p}$), po którego upływie przyrost sygnału
sterującego wynosi zero. W ten sposób zapewnione są własności całkujące układu regulacji predykcyjnej.
\newline Algorytmy MPC cechują się następującymi wymogami i właściwościami:
\begin{itemize}
	\item Wymaganie wyznaczenia wartości przyszłych sygnału sterującego.
	\item Sterowanie według zdefiniowanej trajektorii referencyjnej dla wielkości wyjściowej.
    \item Uwzględnienie przyszłych zmian wartości zadanej. Wcześniejsza reakcja regulatora na 
    przyszłą zmianę wartości referencyjnej kompensuje negatywny wpływ opóźnienia na działanie układu.
	\item Stabilna regulacja obiektów, które nie są minimalnofazowe bez uwzględnienia tego faktu podczas
    syntezy regulatora.
\end{itemize}
Realizację metody sterowania predykcyjnego można zapisać w czterech następujących krokach:
\begin{enumerate}
    \item Pomiar lub estymacja aktualnego stanu obiektu.
    \item Obliczenie przyszłych próbek wyjść systemu.
    \item Zaaplikowanie sygnałów sterujących tylko do następnej chwili czasu.
    \item Powtórzenie algorytmu dla kolejnej chwili czasu.
\end{enumerate}

\section{Model obiektu} \label{sec:model}
Do poprawności działania regulatora MPC niezbędna jest znajomość modelu obiektu, który ma być wysterowany.
Obecnie wykorzystuje się model w postaci równań stanu, podczas gdy w przeszłości korzystano z modelu
odpowiedzi skokowej. Takie podejście wymaga także zaprojektowania obserwatora stanu, używając do tego
metod znanych z teorii sterowania. Model obiektu może być zarówno liniowy, jak i nieliniowy. Jednakże,
użycie modelu nieliniowego prowadzi do nieliniowej optymalizacji, co powoduje zwieloktrotnienie trudności
obliczeniowej. Przekłada się to na zwiększenie wymagań odnośnie częstotliwości taktowania procesora
w implementacji sprzętowej. Wynika z tego stwierdzenie, że modele liniowe mają największe znaczenie
praktyczne z uwagi na możliwość przeprowadzenia obliczeń w czasie rzeczywistym nawet bez wygórowanych
wymagań hardware'owych. Rozwiązaniem tego problemu jest zastosowanie regultaora predykcyjnego w połączeniu
z linearyzacją modelu obiektu w konkretnym punkcie pracy. Następnie wyznaczone są sterowania tak jak dla
liniowego przypadku. Tak zrealizowany algorytm gwarantuje jedynie rozwiązanie suboptymalne, jednak nie
rzutuje to w żaden sposób na przydatność jego realizacji.

\section{Kryterium jakości regulacji} \label{sec:quality}
Jak już wcześniej pokazano w rozdziale \ref{sec:howitworks} w celu wyznaczenia wartości sterowań
w obecnej i następnych chwilach wyznacza się minimum funkcji celu. Funkcja ta określa jakość pracy
regulatora na horyzoncie predykcji. Można stwierdzić, że wartość sygnału sterującego jest wyznaczana
poprzez minimalizację wskaźnika jakości regulacji, który jest inheretnie związany z predykcją wyjścia
obiektu.
\newline W przypadku skalarnym funkcję celu można opisać następującym równaniem:
\begin{equation}
    \begin{dcases}
        J=R_{y}\sum _{i=1}^{N_{p}}(r_{k}-y_{i|k})^{2}+R_{u}\sum _{i=1}^{N_{c}}{(u_{i|k}-u_{k-1})}^{2}\\
        x_{k+1}=Ax_{k}+Bu_{k}\\
        y_{k}=Cx_{k} 
    \end{dcases}
\label{eq:quality}
\end{equation}
\begin{align*}
    x_{k} &= \text{wektor zmiennych stanu w chwili }k\\
    y_{k} &= \text{zmienna wyjściowa w chwili} k\\
    r_{k} &= \text{zmienna referencyjna w chwili} k\\
    u_{k} &= \text{sterowanie w chwili }k\\
    N_{p} &= \text{horyzont predykcji}\\
    N_{c} &= \text{horyzont sterowań}\\
    i|k &= \text{predykcja w chwili }k\text{ odnosząca się do chwili }i\\
    R_{y} &= \text{współczynnik wagowy wyjścia }y\\
    R_{u} &= \text{współczynnik wagowy sterowania}u\\
    A, B, C &= \text{macierze przestrzeni stanu}
\end{align*}

\section{Problem programowania kwadratowego} \label{sec:qp}
\begin{equation}
    J=\frac{1}{2}U^{T}HU+W^{T}U
    \label{eq:J}
\end{equation}
\begin{equation}
	U = \begin{bmatrix}
    u_{k|k}-u_{k-1} \\
	u_{k+1|k}-u_{k-1} \\
    \vdots \\
    u_{k+N_{c}-1|k}-u_{k-1}
	\end{bmatrix}_{N_{c} \times 1}
\label{eq:U}
\end{equation}
\begin{equation}
    H={\phi}^{T}\phi+R_{u}
    \label{eq:H}
\end{equation}
\begin{equation}
    W={\phi}^{T}(R_{s}-Fx_{k})
    \label{eq:W}
\end{equation}
\begin{equation}
    R_{u} = R_{1} 
    \begin{bmatrix}
	    1 & 0 & 0 & \cdots & 0 \\[-0.8ex]
	    0 & 1 & 0 & \cdots & 0 \\[-0.8ex]
	    0 & 0 & 1 & \cdots & 0 \\[-0.8ex]
        \vdots & \vdots & \vdots & \ddots & \vdots \\[-0.8ex]
        0 & 0 & 0 & \cdots & 1
	\end{bmatrix}_{N_{c} \times N_{c}}
\label{eq:Rw}
\end{equation}
\begin{equation}
	\phi = \begin{bmatrix}
	CB & 0 & 0 & \cdots & 0 \\
	CAB & CB & 0 & \cdots & 0 \\
	CA^{2}B & CAB & CB & \cdots & 0 \\
    \vdots & \vdots & \vdots &  & \vdots \\
    CA^{Np-1}B & CA^{Np-2}B & CA^{Np-3}B & \cdots & CA^{Np-Nc}B
	\end{bmatrix}
\label{eq:phi}_{N_{p} \times N_{c}}
\end{equation}
\begin{equation}
	F = \begin{bmatrix}
	CA \\
	CA^{2} \\
	CA^{3} \\
    \vdots \\
    CA^{Np}B
	\end{bmatrix}_{N_{p} \times 1}
\label{eq:F}
\end{equation}
\begin{equation}
    R_{s} = r_{k}\begin{bmatrix}
    1 \\
    1 \\
    \vdots \\
    1
    \end{bmatrix}_{N_{p} \times 1}
\label{eq:Y}
\end{equation}

\section{Pozostałe rodzaje regulatorów klasy MPC} \label{sec:other}

\begin{itemize}
	\item Nonlinear MPC
	\item Explicit MPC
    \item Robust MPC
\end{itemize}

\section{Wady i zalety w porównaniu z regulatorem PID} \label{sec:comparison}
Dorobić tabelę

Porównanie regulacji PID i MPC
Cecha	Regulator PID	Regulator predykcyjny
ograniczenia	brak informacji o ograniczeniach	ograniczenia uwzględnione w projekcie
wartość zadana	wartość zadana daleka od ograniczeń	wartość zadana bliska ograniczeniom
optymalność	sterowanie nie ma charakteru optymalnego	sterowanie ma charakter optymalny
liczba wejść i wyjść układu	jedno wejście i jedno wyjście	wiele wejść i wiele wyjść
model matematyczny	model matematyczny nie jest konieczny

\chapter{Założenia projektowe i~wykorzystane narzędzia}
\section{Założenia projektowe} \label{sec:assumptions}
\begin{itemize}
	\item Układ liniowy, dyskretny
	\item Odpowiedź MCU w czasie rzeczywistym
    \item Poprawność wyliczonych wartości
    \item Opisany przypadek MPC do implementacji -> fast gradient method
    \item Szczególny przypadek regulatora MPC
\end{itemize}

\section{Architektura systemu} \label{sec:system}

\subsection{Procesor - architektura ARM} \label{sec:arm}
ARM (Advanced RISC Machine) jest to rodzina architektur procesorów typu RISC (Reduced Instruction
Set Computing). Charakteryzuje się zmniejszoną liczbą instrukcji w porównaniu do CISC, co
przekłada się na mniejsze zużycie energii. Wynika z tego zastosowanie architektury ARM
w systemach wbudowanych. %więcej dopisać

\subsection{Platforma STM} \label{sec:stm}
STM32 jest to rodzina 32 bitowych mikrokontrolerów produkowana przez STMicroelectronics.
Kontrolery są podzielone na odpowiednie serie, jednak łączy je bazowanie na 32 bitowym
rdzeniu firmy ARM. Grupy te różnią się m.in. częstotliwością taktowania, obsługiwanymi
urządzeniami peryferyjnymi, wsparciem dla arytmetyki zmiennoprzecinkowej, jak i możliwością
cyfrowego przetwarzania sygnałów.
Użyty w projekcie zestaw uruchomieniowy STM32 Nucleo F401RE jest wyposażony w mikrokontroler
STM32F401, który zapewnia wsparcie dla wspomnianej wcześniej jednostki zmiennoprzecinkowej (FPU),
jak również instrukcji cyfrowego przetwarzania sygnałów (DSP). Procesor ten jest oparty na
architekturze ARM Cortex M4. Platforma Nucleo za to dostarcza elastyczne
możliwości budowania oraz projektowania nowych rozwiązań sprzętowych, zarówno
doświadczonym jak i początkującym, użytkownikom. Moduł ten łączy w sobie różne kombinacje
aspektów wydajności oraz zużycia energii. W porównaniu do konkurencyjnych platform oferuje
znacznie mniejsze zużycie energii podczas pracy z zasilaczem impulsowym.
Poniżej zamieszczono pełną specyfikację.

\section{Narzędzia programistyczne} \label{sec:prog}

\subsection{Języki programowania C/C++} \label{sec:cpp}
C jest to proceduralny, strukturalny, statycznie typowany język programowania, który znajduje
zastosowanie w implementacji systemów operacyjnych i wbudowanych. C domyślnie zapewnia
narzędzia, które w sposób efektywny są kompilowane do kodu maszynowego. Taki kod ma porównywalną
wydajność do programów napisanych jedynie za pomocą asemblera. Jest to powiązane ze względną
niskopoziomowością języka C, która zrealizowana została m.in. za pomocą ręcznej alokacji
dynamicznej pamięci, a także minimalnego wspracia w czasie wykonywania programu. 
 
C++ jest to wieloparadygmatowy, statycznie typowany, kompilowalny język programowania zapewniający
znacznie wyższy poziom abstrakcji w porównaniu do języka C. Na potrzeby pracy inżynierskiej
wykorzystano najnowszy, w pełni dostępny, standard tego języka - C++17. 
Wspomniana abstrakcja danych została
zrealizowana za pomocą zastosowania paradygmatu programowania obiektowego. Zastosowanie interfejsów
i "schowanie" implementacji programu umożliwia lepsze przeprowadzenie testów jednostkowych, a także
budowę wieloplatformowych aplikacji. Podejście obiektowe umożliwia ponadto zaprowadzenie znacznie
większego porządku w projektowaniu danego rozwiązania. Jednakże największym atutem takiego 
rozwiązania jest hermetyzacja danych, która zapobiega wprowadzeniu innych wartości do zmiennych
przechowywanych w danej klasie w niepożądanym miejscu programu.
W przypadku programowania rozwiązań przeznaczonych pod współpracę z systemami wbudowanymi użycie
najnowszych możliwości nowszych standardów języka C++ nie jest czasem możliwe. Przykładem jest
w tym przypadku obsługa wyjątków, która pochłania znaczne ilości mocy obliczeniowej, jak
i potrzebnej pamięci. Zagrożeniem jest także używanie Standard Template Library (STL), biblioteki
zawierającej standardowe algorytmy, kontenery i iteratory. Implementacja tejże biblioteki obfita
jest w operacje na stercie, któtych użycie powinno być minimalizowane w świecie systemów
wbudowanych ze względu na fragmentację niewielkiej ilości dostępnej pamięci oraz dłuższy czas
jej alokacji i dealokacji. Ponadto STL zawiera operacje rzucające wcześniej wymienione wyjątki. 

\subsection{Język programowania Python} \label{sec:python}
Wysokopoziomowe ułatwienie testowania z PC.

\subsection{Środowisko MATLAB} \label{sec:matlab}
Sprawdzone środowisko do obliczeń macierzowych i układów dynamicznych.

\subsection{Cube} \label{sec:cube}
Generator kodu do biblioteki HAL.

\subsection{Biblioteka HAL} \label{sec:hal}
W miarę wysokopziomowe rozwiązanie jak na standardy mikroprocków, jednak zostały napotkane problemy.

\subsection{CMake} \label{sec:cmake}
Cross platform make, możliwość kompilacji z różnych systemów operacyjnych.

\subsection{Kompilator i linker} \label{sec:gcc}
arm none eabi gcc, coś o opensource

\subsection{Regex} \label{sec:regex}
Krótko o wyrażeniach regularnych.

\section{Przykład referencyjny} \label{sec:ref}
Ten z simulinka + testowanie w matlabie.

\section{Sposób testowania} \label{sec:tests}
Skrypt w Pythonie + HIL


\chapter{Implementacja rozwiązania}
\section{Ogólny schemat programu} \label{sec:scheme}
Przydałby się rysunek z przepływem informacji + jak działa STM w połączeniu z PC.

\section{Szczegóły implementacji - STM} \label{sec:details-stm}

\subsection{Obiektowość} \label{sec:objects}
C++, singleton, hermetyzacja danych, clean code

\subsection{Komunikacja} \label{sec:uart}
Ramka, schemat, itp

\subsection{Aglorytm} \label{sec:algorithm}
Jak zostało zrealizowane założenie

\section{Szczegóły implementacji - PC} \label{sec:details-pc}
Krótko o skrypcie w Pythonie.

\section{Problemy napotkane podczas realizacji} \label{sec:problems}
HAL - przerwania, ramka UARTa
Model układu - feasibility


\chapter{Przykładowe wyniki}
\section{Różne parametry układu}

\section{Różne parametry regulatora}

\section{Różne wartości zadane}


\chapter{Podsumowanie} \label{ch:summary}
\section{Wyniki}
No działa.

\section{Wnioski}
Jak wyżej.

\section{Pomysły na rozwój projektu}
Jak wyżej.

\begin{appendices}
\end{appendices}

\newpage

% Lists of objects
\listoffigures
\listoftables
\listoflistings

% Bibliography
\nocite{*}
\bibliographystyle{plplain}
\bibliography{references}

\end{document}
