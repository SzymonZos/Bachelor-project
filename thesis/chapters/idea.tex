\section{Wstęp}
Model predictive control (MPC) jest to zaawansowana metoda sterowania, która
polega na takim dobraniu sterowania, aby spełniało ono szereg ograniczeń. Od lat 80. XX wieku
algorytm ten wykorzystywany jest przemyśle procesowym w~zakładach chemicznych i~rafineriach 
ropy naftowej. W ostatnich latach MPC znalazło zastosowanie także w elektrowniach i~elektronice
mocy. Sterowanie predykcyjne wykorzystuje dynamiczny model obiektu, najczęściej jest to empiryczny
model pozyskany za pomocą indetyfikacji systemów. Główną zaletą MPC jest optymalizacja obecnego
przedziału czasowego, biorąc pod uwagę przyszłe stany obiektu. Jest to osiągnięte poprzez
optymalizację skończonego horyzonu czasowego, ale z~wykorzystaniem jedynie sterowania wyliczonego
dla obecnej chwili czasu. Proces ten jest powtarzany z każdą iteracją algorytmu rozwiązującego
układ równań różniczkowych opisujących dany układ. Taki schemat regulacji powoduje, że istnieje
możliwość przewidzenia przyszłych zdarzeń (występujących zgodnie z podanym modelem wartości zadanej)
i podjęcia odpowiednich działań regulujących pracę układem we wcześniejszych chwilach. Sterowanie
predykcyjne jest zazwyczaj zaimplementowane jako dyskretny regulator, lecz obecnie prowadzone są
badania mające na celu uzyskanie szybszej odpowiedzi przy użyciu specjalnie do tego przygotowanych
układów analogowych.

\section{Sposób działania} \label{sec:howitworks}
Zasada pracy regulatora MPC polega na minimalizacji różnic między wartościami predykowanymi:
$y(i+p|i)$ w chwili obecnej $i$ na przyszłą $i+p$, a wartościami zadanymi dla tych wyjść $r(i+p|i)$.
Przez minimalizację tychże różnic rozumiana jest minimalizacja określonego kryterium jakości $J$. W
następnej chwili czasu $(i+1)$ następuje kolejny pomiar sygnału na wyjściu obiektu, a cała procedura
powtarzana jest z takim samym horyzontem predykcji $H(p=1,2,\dots ,H_{max})$. W tym celu stosowana jest
więc zasada sterowania repetycyjnego bazującego na przesuwnym horyzoncie czasu. W algorytmie regulacji MPC
obecny jest także tzw. horyzont sterowania $L$ (gdzie $L \leqslant H$), po którego upływie przyrost sygnału
sterującego wynosi zero. W ten sposób zapewnione są własności całkujące układu regulacji predykcyjnej.
\newline Algorytmy MPC cechują się następującymi wymogami i właściwościami:
\begin{itemize}
	\item Wymaganie wyznaczenia wartości przyszłych sygnału sterującego.
	\item Sterowanie według zdefiniowanej trajektorii referencyjnej dla wielkości wyjściowej.
    \item Uwzględnienie przyszłych zmian wartości zadanej. Wcześniejsza reakcja regulatora na 
    przyszłą zmianę wartości referencyjnej kompensuje negatywny wpływ opóźnienia na działanie układu.
	\item Stabilna regulacja obiektów, które nie są minimalnofazowe bez uwzględnienia tego faktu podczas
    syntezy regulatora.
\end{itemize}
Realizację metody sterowania predykcyjnego można zapisać w czterech następujących krokach:
\begin{enumerate}
    \item Pomiar lub estymacja aktualnego stanu obiektu.
    \item Obliczenie przyszłych próbek wyjść systemu.
    \item Zaaplikowanie sygnałów sterujących tylko do następnej chwili czasu.
    \item Powtórzenie algorytmu dla kolejnej chwili czasu.
\end{enumerate}

\section{Model obiektu} \label{sec:model}
Do poprawności działania regulatora MPC niezbędna jest znajomość modelu obiektu, który ma być wysterowany.
Obecnie wykorzystuje się model w postaci równań stanu, podczas gdy w przeszłości korzystano z modelu
odpowiedzi skokowej. Takie podejście wymaga także zaprojektowania obserwatora stanu, używając do tego
metod znanych z teorii sterowania. Model obiektu może być zarówno liniowy, jak i nieliniowy. Jednakże,
użycie modelu nieliniowego prowadzi do nieliniowej optymalizacji, co powoduje zwieloktrotnienie trudności
obliczeniowej. Przekłada się to na zwiększenie wymagań odnośnie częstotliwości taktowania procesora
w implementacji sprzętowej. Wynika z tego stwierdzenie, że modele liniowe mają największe znaczenie
praktyczne z uwagi na możliwość przeprowadzenia obliczeń w czasie rzeczywistym nawet bez wygórowanych
wymagań hardware'owych. Rozwiązaniem tego problemu jest zastosowanie regultaora predykcyjnego w połączeniu
z linearyzacją modelu obiektu w konkretnym punkcie pracy. Następnie wyznaczone są sterowania tak jak dla
liniowego przypadku. Tak zrealizowany algorytm gwarantuje jedynie rozwiązanie suboptymalne, jednak nie
rzutuje to w żaden sposób na przydatność jego realizacji.

\section{Kryterium jakości regulacji} \label{sec:quality}
Jak już wcześniej pokazano w rozdziale \ref{sec:howitworks} w celu wyznaczenia wartości sterowań
w obecnej i następnych chwilach wyznacza się minimum funkcji celu. Funkcja ta określa jakość pracy
regulatora na horyzoncie predykcji. Można stwierdzić, że wartość sygnału sterującego jest wyznaczana
poprzez minimalizację wskaźnika jakości regulacji, który jest inheretnie związany z predykcją wyjścia
obiektu.
\newline W przypadku skalarnym funkcję celu można opisać następującym równaniem:
\begin{equation}
    J=\sum _{i=1}^{N}w_{x_{i}}(r_{i}-x_{i})^{2}+\sum _{i=1}^{N}w_{u_{i}}{u_{i}}^{2}
\label{eq:quality}
\end{equation}
\begin{align*}
    x_{i} &= i\text{-ta zmienna sterowana}\\
    r_{i} &= i\text{-ta zmienna referencyjna}\\
    u_{i} &= i\text{-ta zmienna sterująca}\\
    w_{x_{i}} &= \text{współczynnik wagowy zmiennej }x_{i}\\
    w_{u_{i}} &= \text{współczynnik wagowy }u_{i}
\end{align*}

\section{Problem programowania kwadratowego} \label{sec:qp}
\begin{equation}
	R = R_{1} * \begin{bmatrix}
	1 & 0 & 0 & 0 \\
	0 & 1 & 0 & 0 \\
	0 & 0 &\ddots & 0 \\
	0 & 0 & 0 & 1
	\end{bmatrix}
\label{eq:hessian}
\end{equation}

\section{Pozostałe rodzaje regulatorów klasy MPC} \label{sec:other}

\begin{itemize}
	\item Nonlinear MPC
	\item Explicit MPC
    \item Robust MPC
\end{itemize}

\section{Wady i zalety w porównaniu z regulatorem PID} \label{sec:comparison}
Dorobić tabelę

Porównanie regulacji PID i MPC
Cecha	Regulator PID	Regulator predykcyjny
ograniczenia	brak informacji o ograniczeniach	ograniczenia uwzględnione w projekcie
wartość zadana	wartość zadana daleka od ograniczeń	wartość zadana bliska ograniczeniom
optymalność	sterowanie nie ma charakteru optymalnego	sterowanie ma charakter optymalny
liczba wejść i wyjść układu	jedno wejście i jedno wyjście	wiele wejść i wiele wyjść
model matematyczny	model matematyczny nie jest konieczny