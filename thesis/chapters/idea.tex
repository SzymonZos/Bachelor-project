\section{Wstęp}
Model predictive control (MPC) jest to zaawansowana metoda sterowania, która
polega na takim dobraniu sterowania, aby spełniało ono szereg ograniczeń. Od lat 80. XX wieku
algorytm ten wykorzystywany jest przemyśle procesowym w~zakładach chemicznych i~rafineriach 
ropy naftowej. W ostatnich latach MPC znalazło zastosowanie także w elektrowniach i~elektronice
mocy. Sterowanie predykcyjne wykorzystuje dynamiczny model obiektu, najczęściej jest to empiryczny
model pozyskany za pomocą indetyfikacji systemów. Główną zaletą MPC jest optymalizacja obecnego
przedziału czasowego, biorąc pod uwagę przyszłe stany obiektu. Jest to osiągnięte poprzez
optymalizację skończonego horyzonu czasowego, ale z~wykorzystaniem jedynie sterowania wyliczonego
dla obecnej chwili czasu. Proces ten jest powtarzany z każdą iteracją algorytmu rozwiązującego
układ równań różniczkowych opisujących dany układ. Taki schemat regulacji powoduje, że istnieje
możliwość przewidzenia przyszłych zdarzeń (występujących zgodnie z podanym modelem wartości zadanej)
i podjęcia odpowiednich działań regulujących pracę układem we wcześniejszych chwilach. Sterowanie
predykcyjne jest zazwyczaj zaimplementowane jako dyskretny regulator, lecz obecnie prowadzone są
badania mające na celu uzyskanie szybszej odpowiedzi przy użyciu specjalnie do tego przygotowanych
układów analogowych.

\section{Sposób działania} \label{sec:howitworks}
Zasada pracy regulatora MPC polega na minimalizacji różnic między wartościami predykowanymi:
\(y(i+p|i)\) w chwili obecnej \(i\) na przyszłą \(i+p\), a wartościami zadanymi dla tych wyjść \(r(i+p|i)\).
Przez minimalizację tychże różnic rozumiana jest minimalizacja określonego kryterium jakości \(J\). W
następnej chwili czasu \((i+1)\) następuje kolejny pomiar sygnału na wyjściu obiektu, a cała procedura
powtarzana jest z takim samym horyzontem predykcji \(H(p=1,2,\dots ,H_{max})\). W tym celu stosowana jest
więc zasada sterowania repetycyjnego bazującego na przesuwnym horyzoncie czasu. W algorytmie regulacji MPC
obecny jest także tzw. horyzont sterowania \(L\) (gdzie \(L<H\)), po którego upływie przyrost sygnału
sterującego wynosi zero. W ten sposób zapewnione są własności całkujące układu regulacji predykcyjnej.
\newline Algorytmy MPC cechują się następującymi wymogami i właściwościami:
\begin{itemize}
	\item Wymaganie wyznaczenia wartości przyszłych sygnału sterującego.
	\item Sterowanie według zdefiniowanej trajektorii referencyjnej dla wielkości wyjściowej.
    \item Uwzględnienie przyszłych zmian wartości zadanej. Wcześniejsza reakcja regulatora na 
    przyszłą zmianę wartości referencyjnej kompensuje negatywny wpływ opóźnienia na działanie układu.
	\item Stabilna regulacja obiektów, które nie są minimalnofazowe bez uwzględnienia tego faktu podczas
    syntezy regulatora.
\end{itemize}
Realizację metody sterowania predykcyjnego można zapisać w czterech następujących krokach:
\begin{enumerate}
    \item Pomiar lub estymacja aktualnego stanu obiektu.
    \item Obliczenie przyszłych próbek wyjść systemu.
    \item Zaaplikowanie sygnałów sterujących tylko do następnej chwili czasu.
    \item Powtórzenie algorytmu dla kolejnej chwili czasu.
\end{enumerate}

\section{Model obiektu} \label{sec:model}
Lorem ipsum.

\section{Kryterium jakości regulacji} \label{sec:quality}
Lorem ipsum.

\section{Problem programowania kwadratowego} \label{sec:qp}
Lorem ipsum.

\section{Pozostałe rodzaje regulatorów klasy MPC} \label{sec:other}

\section{Wady i zalety w porównaniu z regulatorem PID} \label{sec:comparison}
