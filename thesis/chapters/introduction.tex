\section{Motywacja projektu}
Regulacja MPC (Model Predictive Control) jest szybko rozwijającą się częścią automatyki
w ostatnich latach. Jak większość algorytmów regulacji sterowanie predykcyjne powinno
funkcjonować w~wbudowanych systemach czasu rzeczywistego. W~związku z~tym wymagana jest
możliwość pracy przy ograniczonych zasobach sprzętowych.  

\section{Cel pracy}
Celem pracy jest implementacja algorytmu sterowania predykcyjnego na~wybranym
mikrokontrolerze, a~następnie przeprowadzenie weryfikacji jakości wyznaczonych sterowań
oraz otrzymanych odpowiedzi układu regulacji na~zadaną wartość. Na~pełny zakres teorii
regulacji MPC składa się wiele zagadnień. W~badanym przypadku postanowiono ograniczyć
to spektrum do~jednego z~najmniej złożonych przypadków, aby sprawdzić wpływ podstawowych
parametrów regulatora predykcyjnego na~przyjęte kryteria badań. Zakłada się znajomość
modelu matematycznego dyskretnego obiektu regulacji oraz jego liniowy charakter.
