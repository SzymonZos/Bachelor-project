\section{Wyniki}
Zebrane przykładowe wyniki...

\section{Wnioski}
Realizacja postawionych w rozdziale \ref{sec:assumptions} założeń pozwoliła na sprawdzenie
wpływu horyzontu predykcji oraz sterowań na jakość otrzymanej regulacji. Zbyt mały horyzont
predykcji w badanym przypadku skutkował sporą amplitudą kolejnych początkowych sterowań. Zbyt
duży horyzont predykcji wpływał na zwiększenie się opóźnienia w układzie regulacji. Zwiększenie się
horyzontu sterowań było ściśle powiązane z zwiększeniem się złożoności obliczeniowej problemu,
a przede wszystkim konieczności użycia większej ilości pamięci.
Ponadto potwierdzono cechę charakterystyczną sterowania predykcyjnego, a mianowicie minimalizację
wartości pomiędzy kolejnymi sterowaniami. Im horyzont predykcji był większy, tym owa różnica
była mniejsza.

\section{Pomysły na rozwój projektu}
Istnieją dwie płaszczyzny, na których można rozwinąć przygotowany projekt. Pierwsza z nich dotyczy
doboru biblioteki do obsługi rejestrów mikrokontrolera platformy STM. Jedną z bardziej przyjaznych
użytkowniki alternatywnych opcji do korzystania z Cube'a jest otwartoźródłowy projekt
\textit{libopencm3}. Oferowane rozwiązania tejże biblioteki są znacznie bardziej przejrzyste
oraz lepiej udokumentowane. Z jej pomocą można bezproblemowo zastąpić komunikację opartą na
opisanym w rozdziale \ref{sec:uart} pollingu przerwaniami systemowymi. Kolejnym aspektem
stricte programistycznym jest zmiana sposobu alokacji używanych w rozwiązaniu macierzy z
dynamicznego na statyczny. Przeniesienie wykonywania największej części obliczeń ze sterty
na stos poprawiłoby w znaczny sposób wydajność napisanego programu.

Drugą płaszczyzna jest związana z zakresem badanych funkcjonalności regulatora MPC. Warte
rozważenia są w tym przypadku dwie możliwości rozwoju. Pierwsza z nich zakłada rozszerzenie
możliwości pracy regulatora o model nieliniowy przy pomocy techniki linearyzacji. Pozwoliłoby
to zweryfikować poprawność pracy zaimplementowanego algorytmu MPC na szerszej dziedzinie
obiektów. Docelowym rozwiązaniem jest realizacja indetyfikacji obiektu oraz rozwiązywanie
zadania optymalizacji przy nieznanym modelu obiektu. 