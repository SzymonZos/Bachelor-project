\section{Założenia projektowe} \label{sec:assumptions}
Jak pokazano w rozdziale \ref{ch:idea} sterowanie predykcyjne jest bardzo złożonym procesem.
Postanowiono więc przyjąć pewne obostrzenia ułatawiające skupienie się na samym sposobie regulacji, 
wpływie parametrów regulatora oraz obiektu na poprawność działania danego układu regulacji. W tym
celu przyjęto następujące założenie projektowe - model matematyczny badanego układu jest znany, a
także stały w czasie wykonywania się algorytmu. Ponadto obiekt ten jest dyskretnym układem liniowym.
Założenie to pozwoliło na eliminację konieczności przeprowadzenia identyfikacji modelu. Do
implementacji algorytmu MPC wykorzystano szybką metodę gradientową oraz ograniczono się do skalarnej
wersji regulatora.
Podczas oceny poprawności przeprowadzonej symulacji posłużono się następującymi kryteriami:  
\begin{itemize}
	\item Odpowiedź mikrokontrolera w czasie rzeczywistym.
    \item Poprawność wyznaczonych sterowań.
    \item Poprawność otrzymanej wartości wyjściowej.
    \item Uzyskanie wartości zadanej w skończonym czasie.
\end{itemize}

\section{Architektura systemu} \label{sec:system}

\subsection{Procesor - architektura ARM} \label{sec:arm}
ARM (Advanced RISC Machine) jest to rodzina architektur procesorów typu RISC (Reduced Instruction
Set Computing). Charakteryzuje się zmniejszoną liczbą instrukcji w porównaniu do CISC, co
przekłada się na mniejsze zużycie energii. Wynika z tego zastosowanie architektury ARM
w systemach wbudowanych. Dla każdego mikroprocesora 

Control Unit
For any microprocessor, control unit is the heart of the whole process and it is responsible 
for the system operation,so the control unit design is the most important part within the whole
 design. The control unit is sometimes a pure combinational circuit design. Here, the control
  unit is implemented by easy state machine. The processor timing is additionally included within
   the control unit. Signals from the control unit are connected to each component within the 
   processor to supervise its operation.

Arithmetic Logic Unit (ALU)
The ALU has two 32-bits inputs. The primary comes from the register file, whereas the other
 comes from the shifter. Status registers flags modified by the ALU outputs. The V-bit output
  goes to the V flag as well as the Count goes to the C flag. Whereas the foremost significant
   bit really  represents  the S  flag,  the  ALU  output  operation is  done by NORed to  get 
    the  Z  flag. The ALU has a 4-bit function bus that permits up to 16 opcode to be implemented.


\subsection{Platforma STM} \label{sec:stm}
STM32 jest to rodzina 32 bitowych mikrokontrolerów produkowana przez STMicroelectronics.
Kontrolery są podzielone na odpowiednie serie, jednak łączy je bazowanie na 32 bitowym
rdzeniu firmy ARM. Grupy te różnią się m.in. częstotliwością taktowania, obsługiwanymi
urządzeniami peryferyjnymi, wsparciem dla arytmetyki zmiennoprzecinkowej, jak i możliwością
cyfrowego przetwarzania sygnałów.
Użyty w projekcie zestaw uruchomieniowy STM32 Nucleo F401RE jest wyposażony w mikrokontroler
STM32F401, który zapewnia wsparcie dla wspomnianej wcześniej jednostki zmiennoprzecinkowej (FPU),
jak również instrukcji cyfrowego przetwarzania sygnałów (DSP). Procesor ten jest oparty na
architekturze ARM Cortex M4. Platforma Nucleo za to dostarcza elastyczne
możliwości budowania oraz projektowania nowych rozwiązań sprzętowych, zarówno
doświadczonym jak i początkującym, użytkownikom. Moduł ten łączy w sobie różne kombinacje
aspektów wydajności oraz zużycia energii. W porównaniu do konkurencyjnych platform oferuje
znacznie mniejsze zużycie energii podczas pracy z zasilaczem impulsowym.
Poniżej zamieszczono pełną specyfikację.

\section{Narzędzia programistyczne} \label{sec:prog}

\subsection{Języki programowania C/C++} \label{sec:cpp}
C jest to proceduralny, strukturalny, statycznie typowany język programowania, który znajduje
zastosowanie w implementacji systemów operacyjnych i wbudowanych. C domyślnie zapewnia
narzędzia, które w sposób efektywny są kompilowane do kodu maszynowego. Taki kod ma porównywalną
wydajność do programów napisanych jedynie za pomocą asemblera. Jest to powiązane ze względną
niskopoziomowością języka C, która zrealizowana została m.in. za pomocą ręcznej alokacji
dynamicznej pamięci, a także minimalnego wspracia w czasie wykonywania programu. 
Pomimo swoich niewątpliwych zalet istnieją także wady zastosowania tylko i wyłącznie tego
języka. Przykładem niewłaściwych a dopuszczalnych praktyk programistycznych w języku C są
możliwość wywołania niezadeklarowanej funkcji, czy też przyjmowanie przez funkcje zadeklarowane
bez żadnego argumentu dowolnej ilości parametrów. Ponadto legalną operacją jest inicjalizacja
tablicy literałem łańcuchowym składającym się z większej ilości elementów niż jest przeznaczonej
pamięci dla tej zmiennej.

C++ jest to wieloparadygmatowy, statycznie typowany, kompilowalny język programowania zapewniający
znacznie wyższy poziom abstrakcji w porównaniu do języka C. Na potrzeby pracy inżynierskiej
wykorzystano najnowszy, w pełni dostępny, standard tego języka - C++17. 
Wspomniana abstrakcja danych została
zrealizowana za pomocą zastosowania paradygmatu programowania obiektowego. Zastosowanie interfejsów
i 'schowanie' implementacji programu umożliwia lepsze przeprowadzenie testów jednostkowych, a także
budowę wieloplatformowych aplikacji. Podejście obiektowe umożliwia ponadto zaprowadzenie znacznie
większego porządku w projektowaniu danego rozwiązania. Jednakże największym atutem takiego 
rozwiązania jest hermetyzacja danych, która zapobiega wprowadzeniu innych wartości do zmiennych
przechowywanych w danej klasie w niepożądanym miejscu programu.
W przypadku programowania rozwiązań przeznaczonych pod współpracę z systemami wbudowanymi użycie
najnowszych możliwości nowszych standardów języka C++ nie jest czasem możliwe. Przykładem jest
w tym przypadku obsługa wyjątków, która pochłania znaczne ilości mocy obliczeniowej, jak
i potrzebnej pamięci. Zagrożeniem jest także używanie Standard Template Library (STL), biblioteki
zawierającej standardowe algorytmy, kontenery i iteratory. Implementacja tejże biblioteki obfita
jest w operacje na stercie, któtych użycie powinno być minimalizowane w świecie systemów
wbudowanych ze względu na fragmentację niewielkiej ilości dostępnej pamięci oraz dłuższy czas
jej alokacji i dealokacji. Ponadto STL zawiera operacje rzucające wcześniej wymienione wyjątki. 

\subsection{Język programowania Python} \label{sec:python}
Python jest to interpretowany, interaktywny, zoorientowany obiketowo język programowania.
Zawiera on moduły, wyjątki, dynamiczne typowanie, wysokopoziomowe typy danych i klasy.
Python łączy w sobie godną uwagi moc z bardzo czystą składnią. Posiada on interfejs do bardzo
dużej ilości biblioteki i wywołań systemowych. Dodatkowo jest możliwe jego rozszerzenie o własne
biblioteki w C albo C++. Python jest także wykorzystywany jako dodatkowy język przy projektowaniu
aplikacji, które potrzebują programowalnego interfejsu. Warta uwagi jest również przenośność
tego języka pomiędzy popularnymi systemami operacyjnymi takimi jak Linux, Mac, czy też Windows.

W projekcie inżynierkism skorzystano z kilku wysokopoziomowych bibliotek dostępnych
w Pythonie. Do komunikacji z platformą STM została wykorzystana biblioteka \textit{Serial}, która
zawiera odpowiednią implementację wysyłania i odbierania danych za pośrednictwem portu szeregowego.
Przydatna okazała się także biblioteka \textit{NumPy} oferująca klasy macierzy, a także wydajne
obliczenia numeryczne. Zastosowanie znalazła także biblioteka \textit{Pyplot}, która jest
zaopatrzona w funkcje pozwalające graficznie zaprezentować wyniki działania programu. Działanie
dwóch ostatnich bibliotek jest bardzo zbliżone do analogicznych funkcji środowiska MATLAB.

\subsection{Środowisko MATLAB} \label{sec:matlab}
MATLAB (matrix laboratory) jest to wieloparadygmatowe i zamknięte środowisko służące do wykonywania 
obliczeń numerycznych rozwijane przez firmę Mathworks. MATLAB pozwala na bardzo wygodne dla użytkownika
operacje na macierzach, graficzną reprezentację danych w formie wykresów, implementację dużej liczby
algorytmów, a także możliwość stworzenia interfejsu użytkownika. Jednym z pakietów tego środowiska 
jest \textit{Simulink} - graficzne środowisko programistyczne przeznaczone do modelowania, analizy
oraz symulacji systemów dynamicznych. Rozszerzenie to oferuje integrację z resztą modułów MATLABa,
co znacznie upraszcza proces tworzenia modelu. Podstawą funkcjonalności pakietu \textit{Simulink}
są programowalne bloki, które zawierają odpowiednie wywołania funkcji numerycznych. Biblioteka ta
znalazła szerokie zastosowanie w automatyce i cyfrowym przetwarzaniu sygnałów.

\subsection{STM32CubeMX} \label{sec:cube}
STM32CubeMX jest to graficzne narzędzie umożliwiające prostą konfigurację mikrokontrolerów oraz
mikroprocesorów z rodziny STM32. Program ten jest w stanie wygenerować adekwatny system plików
zawierających kod w C przeznaczony do użycia dla rodziny rdzeni Arm Cortex-M. Biblioteka, z której
wywołań korzysta wygenerowany kod ma formę warstwy abstrakcji sprzętowej (HAL). Takie rozwiązanie
umożliwia pominięcie operacji na rejestrach procesora w trakcie tworzenia aplikacji przez użytkownika.

\subsection{CMake} \label{sec:cmake}
CMake jest to otwartoźródłowa, wieloplatformowa rodzina narzędzi przeznaczonych do budowania oraz
testowania oprogramowania. CMake jest używany do kontrolowania kompilacji kodów źródłowych, używając
prostych, niezależnych od platformy i kompilatora plików konfiguracyjnych. Program ten generuje
natywne pliki Makefile, a także przestrzenie robocze, które mogą być użyte w dowolnym zintegrowanym
środowisku programistycznym. 

\subsection{Kompilator i linker} \label{sec:gcc}
Do kompilacji i linkowania wykorzystano zestaw narzędzi programistycznych GNU przeznaczony dla
systemów wbudowanych opartych o architekturę Arm (The GNU Embedded Toolchain Arm), w którego
skład wchodzą m.in. kompilator języka C - arm-none-eabi-gcc, kompilator języka C++
- arm-none-eabi-g++, a także linker - ld. Zdecydowano się na ten zestaw narzędzi ze względu na 
otwartoźródłowość tego projektu.

\subsection{Regex} \label{sec:regex}
Wyrażenia regularne (regex) są to sekwencje znaków, które definiują wzór wyszukiwania. Regex został
rozwinięty jako technika zarówno w teoretycznej informatyce, jak i w teorii języków formalnych.
Z reguły wyrażenia regularne są wykorzystywane w operacjach 'znajdź' lub 'znajdź i zamień', które
zdefiniowane są na łańcuchach znakowych. Innym zastosowaniem ich jest weryfikacja poprawności podanych
danych na wejściu. Istnieją dwie wiodące składnie do zapisu wyrażeń regularnych: pierwsza z nich
wchodzi w skład standardu POSIX, a druga jest wzorowana na oryginalnej implementacji języka Perl. 

\section{Przykład referencyjny} \label{sec:ref}
Do oceny poprawności implementacji rozwiązania wykorzystano 2 przykłady referencyjne, które zostały
przygotowane w \ref{sec:matlab} środowisku MATLAB. Jeden z nich posłużył jako zapoznanie się ze
sposobem regulacji MPC. Został on wykonany w środowisku \textit{Simulink}. 
%Tu potrzebne zdjęcia razem z opisem.
Do weryfikacji poprawności obliczeń przeprowadzonych na platformie STM zaimplementowano analogiczny
alogrytm jako skrypt środowiska MATLAB.  

\section{Sposób testowania} \label{sec:tests}
Testy zostały przeprowadzone jako symulacja Hardware in the loop (HIL). Jest to technika używana
do rozwoju i testowania złożonych wbudowanych systemów czasu rzeczywistego. Symulacja HIL zapewnia
efektywne możliwości przeprowadzenia testów danej platformy poprzez przekazanie kontroli nad 
modelem matematycznym tejże platformie. Właściwości sterowanego modelu są włączane do próby testowej
poprzez dodanie adekwatnego modelu matematycznego reprezentującego opisywany układ dynamiczny. 

Hardware-in-the-loop (HIL) simulation, or HWIL, is a technique that is used in the development
and test of complex real-time embedded systems. HIL simulation provides an effective platform 
by adding the complexity of the plant under control to the test platform. The complexity
of the plant under control is included in test and development by adding a mathematical
representation of all related dynamic systems. These mathematical representations are referred
to as the “plant simulation”. The embedded system to be tested interacts with this plant simulation.
Skrypt w Pythonie + HIL

