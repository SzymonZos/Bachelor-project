\section{Założenia projektowe} \label{sec:assumptions}
\begin{itemize}
	\item Układ liniowy, dyskretny
	\item Odpowiedź MCU w czasie rzeczywistym
    \item Poprawność wyliczonych wartości
    \item Opisany przypadek MPC do implementacji -> fast gradient method
    \item Szczególny przypadek regulatora MPC
\end{itemize}

\section{Architektura systemu} \label{sec:system}

\subsection{Procesor - architektura ARM} \label{sec:arm}
ARM (Advanced RISC Machine) jest to rodzina architektur procesorów typu RISC (Reduced Instruction
Set Computing). Charakteryzuje się zmniejszoną liczbą instrukcji w porównaniu do CISC, co
przekłada się na mniejsze zużycie energii. Wynika z tego zastosowanie architektury ARM
w systemach wbudowanych. %więcej dopisać

\subsection{Platforma STM} \label{sec:stm}
STM32 jest to rodzina 32 bitowych mikrokontrolerów produkowana przez STMicroelectronics.
Kontrolery są podzielone na odpowiednie serie, jednak łączy je bazowanie na 32 bitowym
rdzeniu firmy ARM. Grupy te różnią się m.in. częstotliwością taktowania, obsługiwanymi
urządzeniami peryferyjnymi, wsparciem dla arytmetyki zmiennoprzecinkowej, jak i możliwością
cyfrowego przetwarzania sygnałów.
Użyty w projekcie zestaw uruchomieniowy STM32 Nucleo F401RE jest wyposażony w mikrokontroler
STM32F401, który zapewnia wsparcie dla wspomnianej wcześniej jednostki zmiennoprzecinkowej (FPU),
jak również instrukcji cyfrowego przetwarzania sygnałów (DSP). Procesor ten jest oparty na
architekturze ARM Cortex M4. Platforma Nucleo za to dostarcza elastyczne
możliwości budowania oraz projektowania nowych rozwiązań sprzętowych, zarówno
doświadczonym jak i początkującym, użytkownikom. Moduł ten łączy w sobie różne kombinacje
aspektów wydajności oraz zużycia energii. W porównaniu do konkurencyjnych platform oferuje
znacznie mniejsze zużycie energii podczas pracy z zasilaczem impulsowym.
Poniżej zamieszczono pełną specyfikację.

\section{Narzędzia programistyczne} \label{sec:prog}

\subsection{Języki programowania C/C++} \label{sec:cpp}
C jest to proceduralny, strukturalny, statycznie typowany język programowania, który znajduje
zastosowanie w implementacji systemów operacyjnych i wbudowanych. C domyślnie zapewnia
narzędzia, które w sposób efektywny są kompilowane do kodu maszynowego. Taki kod ma porównywalną
wydajność do programów napisanych jedynie za pomocą asemblera. Jest to powiązane ze względną
niskopoziomowością języka C, która zrealizowana została m.in. za pomocą ręcznej alokacji
dynamicznej pamięci, a także minimalnego wspracia w czasie wykonywania programu. 
 
C++ jest to wieloparadygmatowy, statycznie typowany, kompilowalny język programowania zapewniający
znacznie wyższy poziom abstrakcji w porównaniu do języka C. Na potrzeby pracy inżynierskiej
wykorzystano najnowszy, w pełni dostępny, standard tego języka - C++17. 
Wspomniana abstrakcja danych została
zrealizowana za pomocą zastosowania paradygmatu programowania obiektowego. Zastosowanie interfejsów
i "schowanie" implementacji programu umożliwia lepsze przeprowadzenie testów jednostkowych, a także
budowę wieloplatformowych aplikacji. Podejście obiektowe umożliwia ponadto zaprowadzenie znacznie
większego porządku w projektowaniu danego rozwiązania. Jednakże największym atutem takiego 
rozwiązania jest hermetyzacja danych, która zapobiega wprowadzeniu innych wartości do zmiennych
przechowywanych w danej klasie w niepożądanym miejscu programu.
W przypadku programowania rozwiązań przeznaczonych pod współpracę z systemami wbudowanymi użycie
najnowszych możliwości nowszych standardów języka C++ nie jest czasem możliwe. Przykładem jest
w tym przypadku obsługa wyjątków, która pochłania znaczne ilości mocy obliczeniowej, jak
i potrzebnej pamięci. Zagrożeniem jest także używanie Standard Template Library (STL), biblioteki
zawierającej standardowe algorytmy, kontenery i iteratory. Implementacja tejże biblioteki obfita
jest w operacje na stercie, któtych użycie powinno być minimalizowane w świecie systemów
wbudowanych ze względu na fragmentację niewielkiej ilości dostępnej pamięci oraz dłuższy czas
jej alokacji i dealokacji. Ponadto STL zawiera operacje rzucające wcześniej wymienione wyjątki. 

\subsection{Język programowania Python} \label{sec:python}
Wysokopoziomowe ułatwienie testowania z PC.

\subsection{Środowisko MATLAB} \label{sec:matlab}
Sprawdzone środowisko do obliczeń macierzowych i układów dynamicznych.

\subsection{Cube} \label{sec:cube}
Generator kodu do biblioteki HAL.

\subsection{Biblioteka HAL} \label{sec:hal}
W miarę wysokopziomowe rozwiązanie jak na standardy mikroprocków, jednak zostały napotkane problemy.

\subsection{CMake} \label{sec:cmake}
Cross platform make, możliwość kompilacji z różnych systemów operacyjnych.

\subsection{Kompilator i linker} \label{sec:gcc}
arm none eabi gcc, coś o opensource

\subsection{Regex} \label{sec:regex}
Krótko o wyrażeniach regularnych.

\section{Przykład referencyjny} \label{sec:ref}
Ten z simulinka + testowanie w matlabie.

\section{Sposób testowania} \label{sec:tests}
Skrypt w Pythonie + HIL
