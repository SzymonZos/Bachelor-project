\section{Założenia projektowe} \label{sec:assumptions}
\begin{itemize}
	\item Układ liniowy, deskrytny
	\item Odpowiedź MCU w czasie rzeczywistym
    \item Poprawność wyliczonych wartości
    \item Opisany przypadek MPC do implementacji -> fast gradient method
    \item Szczególny przypadek regulatora MPC
\end{itemize}

\section{Architektura systemu} \label{sec:system}

\subsection{Platforma STM} \label{sec:stm}
STM32 jest to rodzina 32 bitowych mikrokontrolerów produkowana przez STMicroelectronics.
Kontrolery są podzielone na odpowiednie serie, jednak łączy je bazowanie na 32 bitowym
rdzeniu firmy ARM. Grupy te różnią się m.in. częstotliwością taktowania, obsługiwanymi
urządzeniami peryferyjnymi, wsparciem dla arytmetyki zmiennoprzecinkowej, jak i możliwością
cyfrowego przetwarzania sygnałów.
Użyty w projekcie zestaw uruchomieniowy STM32 Nucleo F401RE zapewnia elastyczne
możliwości budowania oraz projektowania nowych rozwiązań sprzętowych, zarówno
doświadczonym jak i początkującym, użytkownikom. Moduł ten łączy w sobie różne kombinacje
aspektów wydajności i zużycia energii dostarczone przez mikrokontroler STM32F401.
W porównaniu do konkurencyjnych platform oferuje znacznie mniejsze zużycie energii
podczas pracy z zasilaczem impulsowym.
STM32F4 series of high-performance MCUs with DSP and FPU instructions

\subsection{Procesor - architektura ARM} \label{sec:arm}
ARM, previously Advanced RISC Machine, originally Acorn RISC Machine, is a family of reduced
instruction set computing (RISC) architectures for computer processors, configured for various
environments.

\section{Narzędzia programistyczne} \label{sec:prog}

\subsection{Języki programowania C/C++} \label{sec:cpp}
Paradygmaty, co, jak i dlaczego.

\subsection{Język programowania Python} \label{sec:python}
Wysokopoziomowe ułatwienie testowania z PC.

\subsection{Środowisko MATLAB} \label{sec:matlab}
Sprawdzone środowisko do obliczeń macierzowych i układów dynamicznych.

\subsection{Cube} \label{sec:cube}
Generator kodu do biblioteki HAL.

\subsection{Biblioteka HAL} \label{sec:hal}
W miarę wysokopziomowe rozwiązanie jak na standardy mikroprocków, jednak zostały napotkane problemy.

\subsection{CMake} \label{sec:cmake}
Cross platform make, możliwość kompilacji z różnych systemów operacyjnych.

\subsection{Kompilator i linker} \label{sec:gcc}
arm none eabi gcc, coś o opensource

\subsection{Regex} \label{sec:regex}
Krótko o wyrażeniach regularnych.

\section{Przykład referencyjny} \label{sec:ref}
Ten z simulinka + testowanie w matlabie.

\section{Sposób testowania} \label{sec:tests}
Skrypt w Pythonie
