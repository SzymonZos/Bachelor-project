\section{Ogólny schemat programu} \label{sec:scheme}
Przydałby się rysunek z przepływem informacji + jak działa STM w połączeniu z PC.

\section{Szczegóły implementacji - STM} \label{sec:details-stm}

\subsection{Obiektowość} \label{sec:objects}
C++, singleton, hermetyzacja danych, clean code
\begin{listing}[htb]
\begin{minted}{cpp}
#ifndef MPC_STM_DATAPARSER_HPP
#define MPC_STM_DATAPARSER_HPP

#include <map>
#include <vector>
#include <string>

namespace Utils {
    class DataParser {
    public:
        static bool isNewDataGoingToBeSend;

        DataParser() = default;
        ~DataParser() = default;
        void ParseReceivedMsg(const std::string& msg);

        [[nodiscard]] std::map<std::string,
        std::vector<double>> GetStorage() const;
        
        void ClearStorage();

    private:
        constexpr static const char* regexPattern = 
        R"(([[:alpha:]]+)(': )(\[.+?\]))";

        std::map<std::string, std::vector<double>> storage;
    };
}

#endif //MPC_STM_DATAPARSER_HPP

\end{minted}
\caption{DataParser.hpp: Przykładowy plik nagłówkowy}
\label{lst:header_example`}
\end{listing}

\begin{listing}[htb]
\begin{minted}{cpp}
#include "utils/DataParser.hpp"
#include "utils/Misc.hpp"
#include <regex>
#include <algorithm>


bool Utils::DataParser::isNewDataGoingToBeSend = false;


std::map<std::string, std::vector<double>> Utils::DataParser::GetStorage() const {
    return storage;
}


void Utils::DataParser::ParseReceivedMsg(const std::string& msg) {
    std::string valuesMatch;
    std::regex pattern(regexPattern);
    std::smatch fullMatch;
    std::string::const_iterator iterator(msg.cbegin());

    while (std::regex_search(iterator, msg.cend(), fullMatch, pattern)) {
        valuesMatch = fullMatch[3].str();
        std::replace(valuesMatch.begin(), valuesMatch.end(), ',', ' ');
        valuesMatch.pop_back(); // trim ]
        valuesMatch.erase(0, 1); // trim [
        if (fullMatch[1].str().find('A') != std::string::npos) {
            Utils::Misc::StringToDouble(valuesMatch, storage["A"]);
        } else if (fullMatch[1].str().find('B') != std::string::npos) {
            Utils::Misc::StringToDouble(valuesMatch, storage["B"]);
        } else if (fullMatch[1].str().find('C') != std::string::npos) {
            Utils::Misc::StringToDouble(valuesMatch, storage["C"]);
        } else if (fullMatch[1].str().find("set") != std::string::npos) {
            Utils::Misc::StringToDouble(valuesMatch, storage["set"]);
        } else if (fullMatch[1].str().find("control") != std::string::npos) {
            Utils::Misc::StringToDouble(valuesMatch, storage["control"]);
        } else if (fullMatch[1].str().find("horizon") != std::string::npos) {
            Utils::Misc::StringToDouble(valuesMatch, storage["horizons"]);
        }
        iterator = fullMatch.suffix().first;
    }
}


void Utils::DataParser::ClearStorage() {
    storage.clear();
}

\end{minted}
\caption{DataParser.cpp: Przykładowy plik źródłowy}
\label{lst:source_example`}
\end{listing}

\begin{listing}[htb]
\begin{minted}{cpp}
namespace HAL {
    class Peripherals {
    public:
        Peripherals(const Peripherals&) = delete;
        Peripherals& operator=(const Peripherals&) = delete;
        Peripherals(Peripherals&&) = delete;
        Peripherals& operator=(Peripherals&&) = delete;

        static Peripherals& GetInstance();
    };
}
\end{minted}
\caption{Peripherals.hpp: Wzorzec projektowy - singleton}
\label{lst:singleton_hpp`}
\end{listing}

\begin{listing}[htb]
\begin{minted}{cpp}
HAL::Peripherals& HAL::Peripherals::GetInstance() {
    static Peripherals instance;
    return instance;
}
\end{minted}
\caption{Peripherals.cpp: Wzorzec projektowy - singleton}
\label{lst:singleton_cpp}
\end{listing}

\subsection{Komunikacja} \label{sec:uart}
Ramka, schemat, itp
\begin{listing}[htb]
\begin{minted}{cpp}
void HAL_GPIO_EXTI_Callback(uint16_t GPIO_Pin) {
    if(GPIO_Pin == B1_Pin) {
        HAL_GPIO_TogglePin(GPIOA, GPIO_PIN_5);
        Utils::DataParser::isNewDataGoingToBeSend = true;
    }
}
\end{minted}
\caption{Peripherals.cpp: Implementacja wywołania zwrotnego (callback) podczas przerwania}
\label{lst:callback_cpp}
\end{listing}

\begin{listing}[htb]
\begin{minted}{cpp}
    void HAL::Peripherals::SendString(const uint8_t* string, uint16_t timeout) {
        HAL_UART_Transmit(&handleUART, const_cast<uint8_t*>(string),
        std::strlen(reinterpret_cast<const char*>(string)), timeout);
    }
    \end{minted}
    \caption{Peripherals.cpp: Implementacja wysyłania danych z platformy STM do PC}
    \label{lst:sendstring_cpp}
\end{listing}

\begin{listing}[htb]
\begin{minted}{cpp}
void HAL::Peripherals::ReceiveString(const uint8_t* string, uint16_t timeout) {
    uint8_t bufSize = 0;
    HAL_UART_Receive(&handleUART, &bufSize, 1, timeout);
    HAL_UART_Receive(&handleUART, const_cast<uint8_t*>(string), bufSize, 100);
}
\end{minted}
\caption{Peripherals.cpp: Implementacja odbierania danych przez platformę STM}
\label{lst:sendstring_cpp}
\end{listing}

\subsection{Klasa macierzy} \label{sec:matrices}

\begin{listing}[htb]
\begin{minted}{cpp}
#include ...

class CMatrix {
protected:
    uint32_t rows;
    uint32_t columns;
    double** matrix{};
    void make_matrix();

public:
    CMatrix();
    CMatrix(uint32_t rows, uint32_t columns);
    CMatrix(uint32_t rows, uint32_t columns, double** mat);
    CMatrix(uint32_t rows, uint32_t columns, const double* mat);
    explicit CMatrix(uint32_t rows, const std::string& value = "");
    CMatrix(const CMatrix& M);
    ~CMatrix();

    [[nodiscard]] uint32_t GetRows() const { return rows; }
    [[nodiscard]] uint32_t GetColumns() const { return columns; }
    void GetSize(uint32_t& rows, uint32_t& columns) const;
    [[nodiscard]] double& GetElement(uint32_t row, uint32_t column) const;
    void SetRow(uint32_t row, const CMatrix& v);
    void SetColumn(uint32_t column, const CMatrix& v);
    void SetValue(double value);
    [[nodiscard]] double Det() const;
    [[nodiscard]] CMatrix Inverse() const;

    double* operator[](uint32_t row) const;
    CMatrix& operator= (const CMatrix& m);
    CMatrix operator+ (const CMatrix& m) const;
    CMatrix& operator+= (const CMatrix& m);
    CMatrix operator- (const CMatrix& m) const;
    CMatrix& operator-= (const CMatrix& m);
    CMatrix operator* (const CMatrix& m) const;
    CMatrix& operator*= (const CMatrix& m);
    CMatrix operator* (const double& scalar) const;
    CMatrix& operator*= (const double& scalar);
    CMatrix operator/ (const double& scalar) const;
    CMatrix& operator/= (const double& scalar);
    CMatrix operator^ (const uint32_t& exponent) const;
    CMatrix operator-() const;
    [[nodiscard]] CMatrix T() const;
};
\end{minted}
\caption{Matrix.hpp: Fragment pliku nagłówkowego zawierającego własną implementację macierzy}
\label{lst:matrix_hpp}
\end{listing}

\subsection{Aglorytm} \label{sec:algorithm}
Jak zostało zrealizowane założenie

\section{Szczegóły implementacji - PC} \label{sec:details-pc}
Krótko o skrypcie w Pythonie.

\section{Problemy napotkane podczas realizacji} \label{sec:problems}
HAL - przerwania, ramka UARTa
Model układu - feasibility
