\section{Ogólny schemat programu} \label{sec:scheme}
Przydałby się rysunek z przepływem informacji + jak działa STM w połączeniu z PC.

\section{Szczegóły implementacji - STM} \label{sec:details-stm}
Zdecydowano się na implementację części programowej w języku C++, zważywszy na jego benefity
omówione w rozdziale \ref{sec:cpp}. W kolejnych podrozdziałach zaprezentowano szczegóły
zaprogramowanego rozwiązania problemu. Omówiono także na przykładach zastosowane praktyki
programistyczne wraz z możliwością ich rozwoju oraz usprawnienia działania. 

\subsection{Charakterystyczne cechy C++} \label{sec:cpp_details}
W projekcie wykorzystano paradygmat programowania obiektowego ze względu na uproszczenie zarządzania
dostępem do danych przez konkretne funkcje, a także w celu uporządkowania konkretnych rozwiązań.
Dodatkowo takie podejście w łatwy sposób można rozszerzyć o nowe funkcjonalności. Nieodłącznym 
aspektem programistycznym są testy jednostkowe, ponieważ pozwalają one na wczesne wykrycie błędów
danego oprogramowania. Implementacja obiektowości znacznie ułatwia testowanie kodu napisanego w C++.
W tym celu można posłużyć się np. otwartoźródłową biblioteką Googletest. Na listingu
\ref{lst:header_example} zaprezentowano przykładowy plik nagłówkowy jednej z klas użytych w
implementacji rozwiązania. Podziałem na pola i metody publiczne oraz prywatne zrealizowano
postulat hermetyzacji danych. Do pola \textit{storage} spoza klasy można odwołać się tylko poprzez
adekwatną metodę \textit{Get}. Dzięki takiemu rozwiązaniu uniknięto niepożądanych zmian zawartości
tego pola poprzez metody nienależące do klasy \textit{DataParser}. Dodatkową organizację i porządek
w zrealizowanym projekcie zapewniło użycie przestrzeni nazw (namespace). 
\begin{listing}[htb]
\begin{minted}[linenos, breaklines]{cpp}
#ifndef MPC_STM_DATAPARSER_HPP
#define MPC_STM_DATAPARSER_HPP

#include <map>
#include <vector>
#include <string>

namespace Utils {
    class DataParser {
    public:
        static bool isNewDataGoingToBeSend;

        DataParser() = default;
        ~DataParser() = default;
        void ParseReceivedMsg(const std::string& msg);

        [[nodiscard]] std::map<std::string,
        std::vector<double>> GetStorage() const;
        
        void ClearStorage();

    private:
        constexpr static const char* regexPattern = 
        R"(([[:alpha:]]+)(': )(\[.+?\]))";

        std::map<std::string, std::vector<double>> storage;
    };
}

#endif //MPC_STM_DATAPARSER_HPP
\end{minted}
\caption{DataParser.hpp: Przykładowy plik nagłówkowy}
\label{lst:header_example}
\end{listing}

Na listingu \ref{lst:source_example} przedstawiono plik źródłowy korespondujący z plikiem
nagłówkowym klasy \textit{DataParser}. Rozdzielenie deklaracji i definicji pól oraz metod umożliwia
odwołanie się do tych publicznych poza zakresem danej klasy poprzez włączenie odpowiedniego 
pliku nagłówkowego. W celu zapewnienia bardziej elastycznego sposobu identyfikacji odebranych
danych posłużono się omówionymi w rozdziale \ref{sec:regex} wyrażeniami regularnymi. Zastosowanie
wyszukiwania kluczowych wartości w przeprowadzaniu analizy odebranej wiadomości umożliwia
przyszłą modyfikację tychże komunikatów. Aby uprościć implementację biblioteki \textit{<regex>},
zdecydowano się na wykorzystanie STLa. Stało się to możliwe dzięki pokaźnemu, jak na systemy
wbudowane, rozmiarowi pamięci flash platformy Nucleo. Całość została zrealizowana jako maszyna stanów.

\begin{listing}[p]
\centering
\begin{minted}[linenos, breaklines]{cpp}
#include ...

bool Utils::DataParser::isNewDataGoingToBeSend = false;

std::map<std::string, std::vector<double>> Utils::DataParser::GetStorage() const {
    return storage;
}

void Utils::DataParser::ParseReceivedMsg(const std::string& msg) {
    std::string valuesMatch;
    std::regex pattern(regexPattern);
    std::smatch fullMatch;
    std::string::const_iterator iterator(msg.cbegin());

    while (std::regex_search(iterator, msg.cend(), fullMatch, pattern)) {
        valuesMatch = fullMatch[3].str();
        std::replace(valuesMatch.begin(), valuesMatch.end(), ',', ' ');
        valuesMatch.pop_back(); // trim ]
        valuesMatch.erase(0, 1); // trim [
        if (fullMatch[1].str().find('A') != std::string::npos) {
            Utils::Misc::StringToDouble(valuesMatch, storage["A"]);
        } else if (fullMatch[1].str().find('B') != std::string::npos) {
            Utils::Misc::StringToDouble(valuesMatch, storage["B"]);
        } else if (fullMatch[1].str().find('C') != std::string::npos) {
            Utils::Misc::StringToDouble(valuesMatch, storage["C"]);
        } else if (fullMatch[1].str().find("set") != std::string::npos) {
            Utils::Misc::StringToDouble(valuesMatch, storage["set"]);
        } else if (fullMatch[1].str().find("control") != std::string::npos) {
            Utils::Misc::StringToDouble(valuesMatch, storage["control"]);
        } else if (fullMatch[1].str().find("horizon") != std::string::npos) {
            Utils::Misc::StringToDouble(valuesMatch, storage["horizons"]);
        }
        iterator = fullMatch.suffix().first;
    }
}

void Utils::DataParser::ClearStorage() {
    storage.clear();
}
\end{minted}
\caption{DataParser.cpp: Przykładowy plik źródłowy}
\label{lst:source_example}
\end{listing}

W celu lepszego dostosowania aplikacji do standardu języka C++ postanowiono zastosować jeden z wzorców
projektowych, a mianowicie Singleton. Przykład implementacji pokazano na listingach \ref{lst:singleton_hpp}
oraz \ref{lst:singleton_cpp}. Zdecydowano się na statyczną wersję tego wzorca ze względu na bezpieczeństwo
w poprawnym funkcjonowaniu procesów. Rozwiązanie proponowane za pośrednictwem gotowego przykładu użycia
interferjsu biblioteki HAL jest bardzo niebezpieczne z punktu widzenia dostępu do danych. Mianowicie, po
inicjalizacji urządzeń peryferyjnych korzysta się w owym przykładzie z jednej struktury, która jest zmienną
globalną. Przykład ten pokazuje, że rozwiązania tej konkretnej biblioteki nie należą do optymalnych.
Z tego powodu postanowiono stworzyć osobną klasę, która powinna być inicjalizowana tylko raz. W takim
rozwiązaniu problemu Singleton jest dobrym pomysłem na poprawę jakości kodu.

\begin{listing}[htb]
\begin{minted}[linenos, breaklines]{cpp}
namespace HAL {
    class Peripherals {
    public:
        Peripherals(const Peripherals&) = delete;
        Peripherals& operator=(const Peripherals&) = delete;
        Peripherals(Peripherals&&) = delete;
        Peripherals& operator=(Peripherals&&) = delete;

        static Peripherals& GetInstance();
    };
}
\end{minted}
\caption{Peripherals.hpp: Wzorzec projektowy - singleton}
\label{lst:singleton_hpp}
\end{listing}

\begin{listing}[htb]
\begin{minted}[linenos, breaklines]{cpp}
HAL::Peripherals& HAL::Peripherals::GetInstance() {
    static Peripherals instance;
    return instance;
}
\end{minted}
\caption{Peripherals.cpp: Wzorzec projektowy - singleton}
\label{lst:singleton_cpp}
\end{listing}

\subsection{Komunikacja} \label{sec:uart}
W implementacji komunikacji po stronie platformy STM wykorzystano uniwersalny asynchroniczny
nadajnik-odbiornik (UART). Jest to układ scalony, który wykorzystuje się w celu odbierania i przesyłania
informacji poprzez port szeregowy. W tym celu wykorzystano gotowe, wysokopoziomowe rozwiązania
biblioteki HAL. Aby możliwe było wysłanie nowych parametrów regulatora oraz układu wykorzystano wspomniany
w rozdziale \ref{sec:stm} programowalny przycisk dostępny dla użytkownika. Postanowiono zastosować do tego
celu jedno z systemowych przerwań wraz z wywołaniem zwrotnym widocznym na listingu \ref{lst:callback_cpp}.
Taki sposób rozwiązania problemu pozwolił na wyeliminowanie zjawiska odpytywania (pollingu). W oczekiwaniu
na interakcję użytkownika, procesor systematycznie nie sprawdza danego rejestru.
\begin{listing}[htb]
\begin{minted}[linenos, breaklines]{cpp}
void HAL_GPIO_EXTI_Callback(uint16_t GPIO_Pin) {
    if(GPIO_Pin == B1_Pin) {
        HAL_GPIO_TogglePin(GPIOA, GPIO_PIN_5);
        Utils::DataParser::isNewDataGoingToBeSend = true;
    }
}
\end{minted}
\caption{Peripherals.cpp: Implementacja wywołania zwrotnego (callback) podczas przerwania}
\label{lst:callback_cpp}
\end{listing}

W przypadku implementacji funkcji stricte odpowiedzialnych za komunikację posłużono się wcześniej
wspomnianym pollingiem. Skorzystanie z obsługi przerwań w tym przypadku nie było możliwe zważywszy na
niewłaściwą interakcję między tym generowanym po wciśnięciu przecisku a tymi po wysłaniu czy też odebraniu
wiadomości. Takie podejście prowadzi do nadmiernego zużycia zasobów procesora, jednakże umożliwiło ono 
tymczasowe rozwiązanie napotkanego problemu. Zgodnie z przyjętym paradygmatem programowania obiektowego
napisano metody opakowujące biblioteczne funkcje obsługujące UARTa. Wspomniane funkcjonalności
zaprezentowano na listingach \ref{lst:sendstring_cpp} oraz \ref{lst:receivestring_cpp}. Implementacja
wysyłania wiadomości okazała się prostolinijna w przeciwieństwie do odbioru wiadomości. Problemu
przysporzył brak możliwości skutecznego nawiązania połączenia bez znajomości długości przychodzącego
łańcucha znaków. Postanowiono więc wysyłać dwie wiadomości - pierwszą o z góry ustalonym rozmiarze
1 bajta zawierającą informację o długości następnej. Takie ograniczenie poskutkowało zmniejszeniem
maksymalnej pojemności wiadomości do 255 bajtów, co jest wystarczającą wielkością do użycia w celach
projektowych. Istnieje możliwość zwiększenia tego limitu poprzez ustalenie innej liczby bajtów w pierwszej
przychodzącej wiadomości. 
\begin{listing}[htb]
\begin{minted}[linenos, breaklines]{cpp}
    void HAL::Peripherals::SendString(const uint8_t* string, uint16_t timeout) {
        HAL_UART_Transmit(&handleUART, const_cast<uint8_t*>(string),
        std::strlen(reinterpret_cast<const char*>(string)), timeout);
    }
    \end{minted}
    \caption{Peripherals.cpp: Implementacja wysyłania danych z platformy STM do PC}
    \label{lst:sendstring_cpp}
\end{listing}

\begin{listing}[htb]
\begin{minted}[linenos, breaklines]{cpp}
void HAL::Peripherals::ReceiveString(const uint8_t* string, uint16_t timeout) {
    uint8_t bufSize = 0;
    HAL_UART_Receive(&handleUART, &bufSize, 1, timeout);
    HAL_UART_Receive(&handleUART, const_cast<uint8_t*>(string), bufSize, 100);
}
\end{minted}
\caption{Peripherals.cpp: Implementacja odbierania danych przez platformę STM}
\label{lst:receivestring_cpp}
\end{listing}

\subsection{Klasa macierzy} \label{sec:matrices}

\begin{listing}[htb]
\begin{minted}[linenos, breaklines]{cpp}
#include ...

class CMatrix {
protected:
    uint32_t rows;
    uint32_t columns;
    double** matrix{};
    void make_matrix();

public:
    CMatrix();
    CMatrix(uint32_t rows, uint32_t columns);
    CMatrix(uint32_t rows, uint32_t columns, double** mat);
    CMatrix(uint32_t rows, uint32_t columns, const double* mat);
    explicit CMatrix(uint32_t rows, const std::string& value = "");
    CMatrix(const CMatrix& M);
    ~CMatrix();

    [[nodiscard]] uint32_t GetRows() const { return rows; }
    [[nodiscard]] uint32_t GetColumns() const { return columns; }
    void GetSize(uint32_t& rows, uint32_t& columns) const;
    [[nodiscard]] double& GetElement(uint32_t row, uint32_t column) const;
    void SetRow(uint32_t row, const CMatrix& v);
    void SetColumn(uint32_t column, const CMatrix& v);
    void SetValue(double value);
    [[nodiscard]] double Det() const;
    [[nodiscard]] CMatrix Inverse() const;

    double* operator[](uint32_t row) const;
    CMatrix& operator= (const CMatrix& m);
    CMatrix operator+ (const CMatrix& m) const;
    CMatrix& operator+= (const CMatrix& m);
    CMatrix operator- (const CMatrix& m) const;
    CMatrix& operator-= (const CMatrix& m);
    CMatrix operator* (const CMatrix& m) const;
    CMatrix& operator*= (const CMatrix& m);
    CMatrix operator* (const double& scalar) const;
    CMatrix& operator*= (const double& scalar);
    CMatrix operator/ (const double& scalar) const;
    CMatrix& operator/= (const double& scalar);
    CMatrix operator^ (const uint32_t& exponent) const;
    CMatrix operator-() const;
    [[nodiscard]] CMatrix T() const;
};
\end{minted}
\caption{Matrix.hpp: Fragment pliku nagłówkowego zawierającego własną implementację macierzy}
\label{lst:matrix_hpp}
\end{listing}

\subsection{Aglorytm} \label{sec:algorithm}
\begin{listing}[htb]
\begin{minted}[linenos, breaklines]{cpp}
double Control::MPC::FastGradientMethod(const std::string& msg) {
    CMatrix J(1,1), J_prev(1,1);
    Utils::Misc::StringToDouble(msg, sys.x);
    for (uint32_t i = 0; i < 100; i++) {
        opt.W = opt.fi.T() * ((opt.F * sys.x) - opt.Rs);
        CalculateProjectedGradientStep();
        J_prev = J;
        J = sys.v.T() * opt.H * sys.v / 2 + sys.v.T() * opt.W;
        if (std::fabs(J_prev[0][0] - J[0][0]) < eps) {
            break;
        }
    }
    return sys.v[0][0];
}
\end{minted}
\caption{MPC.cpp: Implementacja szybkiej metody gradientowej}
\label{lst:gradient_cpp}
\end{listing}

\begin{listing}[htb]
\begin{minted}[linenos, breaklines]{cpp}
void Control::MPC::CalculateProjectedGradientStep() {
    CVector gradient(sys.v.GetRows(), 1), v_old = sys.v;
    gradient = opt.H * sys.v + opt.W;
    sys.w = sys.v - gradient * eigenvalues.step;
    for (uint32_t i = 0; i < sys.v.GetRows(); i++) {
        if (sys.w[i][0] < controlValues.min) {
            sys.v[i][0] = controlValues.min;
        }
        else if (sys.w[i][0] > controlValues.max) {
            sys.v[i][0] = controlValues.max;
        } else {
            sys.v[i][0] = sys.w[i][0];
        }
    }
    sys.w = sys.v + (sys.v - v_old) * eigenvalues.fastConvergence;
}
\end{minted}
\caption{MPC.cpp: Implementacja obliczenia kolejnego kroku w algorytmie gradientowym}
\label{lst:step_cpp}
\end{listing}

\begin{listing}[htb]
\begin{minted}[linenos, breaklines]{cpp}
void Control::MPC::InitializeParameters(std::map<std::string, std::vector<double>> storage) {
    uint32_t dimension = static_cast<uint32_t>(storage["C"].size());
    controlValues.min = storage["control"][0];
    controlValues.max = storage["control"][1];

    horizons.prediction = storage["horizons"][0];
    horizons.control = storage["horizons"][1];

    sys.A(dimension, dimension, storage["A"].data());
    sys.B(dimension, 1, storage["B"].data());
    sys.C(1, dimension, storage["C"].data());
    sys.x(dimension, 1);
    sys.v(horizons.control, 1);
    sys.w(horizons.control, 1);

    opt.F(horizons.prediction, dimension);
    opt.fi(horizons.prediction, horizons.control);
    opt.Rw(horizons.control, horizons.control, "eye");
    opt.Rs(horizons.prediction, 1);
    opt.H(horizons.control, horizons.control);
    opt.W(horizons.control, 1);
    opt.scalarRs = storage["set"][0];
    opt.scalarRw = 1;

    CalculateOptimizationMatrices();

    eigenvalues.max = Utils::Misc::PowerMethod(opt.H, 20);
    eigenvalues.min = Utils::Misc::PowerMethod(opt.H.Inverse(), 20);
    eigenvalues.step = 1 / eigenvalues.max;
    eigenvalues.fastConvergence = (std::sqrt(eigenvalues.max) - std::sqrt(eigenvalues.min)) /
                                  (std::sqrt(eigenvalues.max) + std::sqrt(eigenvalues.min));
}
\end{minted}
\caption{MPC.cpp: Implementacja zadawania nowych parametrów}
\label{lst:parameters_cpp}
\end{listing}

\begin{listing}[htb]
\begin{minted}[linenos, breaklines]{cpp}
void Control::MPC::CalculateOptimizationMatrices() {
    opt.Rw *= opt.scalarRw;
    opt.Rs.SetValue(opt.scalarRs);
    CVector productMatrix(sys.C.GetColumns());
    productMatrix = sys.C * (sys.A ^ (horizons.prediction - horizons.control));
    for (uint32_t i = horizons.prediction; i != 0; i--) {
        for (uint32_t j = horizons.control; j != 0; j--) {
            if (i == horizons.prediction) {
                if (j < horizons.control) {
                    productMatrix *= sys.A;
                }
                opt.fi[i-1][j-1] = (productMatrix * sys.B)[0][0];
            } else if (i < j && j < horizons.control) {
                opt.fi[i-1][j-1] = opt.fi[i][j];
            }
        }
    }
    opt.H = opt.fi.T() * opt.fi + opt.Rw;
    productMatrix = sys.C * sys.A;
    for (uint32_t i = 0; i < horizons.prediction; i++) {
        opt.F.SetRow(i, productMatrix);
        productMatrix *= sys.A;
    }
}
\end{minted}
\caption{MPC.cpp: Implementacja wyliczenia nowych macierzy optymalizacyjnych}
\label{lst:opt_cpp}
\end{listing}

Jak zostało zrealizowane założenie

\section{Szczegóły implementacji - PC} \label{sec:details-pc}
\begin{listing}[htb]
\begin{minted}[linenos, breaklines]{python}
projectPath = os.path.dirname(os.path.abspath(__file__))
changeParameters = True
systemParameters = {'A': [1, 0, 1, 0, 0, 1, 0, 1, 0, 0, 1, 0, 0, 0, 0, 1],
                    'B': [0, 0, 1, 0],
                    'C': [1, 1, 0, 0],
                    'setPoint': [10],
                    'controlExtremeValues': [-10, 10],
                    'horizons': [30, 8]}
systemParametersToSend = str(systemParameters)[1:-1] + '\n\0'
A = np.array(systemParameters['A']).\
    reshape(-1, int(np.sqrt(len(systemParameters['A']))))
B = np.array(systemParameters['B']).\
    reshape(len(systemParameters['B']), -1)
C = np.array(systemParameters['C'])
x = np.zeros((len(systemParameters['B']), 1))
y = []
u = []
timer = []
\end{minted}
\caption{Inicjalizacja danych w skrypcie Pythona}
\label{lst:init_py}
\end{listing}

\begin{listing}[htb]
\begin{minted}[linenos, breaklines]{python}
with serial.Serial('COM3', 115200, timeout=20) as ser:
    if changeParameters:
        ser.write(bytes([len(systemParametersToSend)]))
        ser.write(systemParametersToSend.encode())
        startTimer = time.time()
        print(ser.readline().decode('utf-8'))
        endTimer = time.time()
        initTimer = endTimer - startTimer
    for i in range(200):
        xToSend = np.array2string(
            x.flatten(), formatter={'float_kind': lambda number: "%.4f" % number})
        xToSend = xToSend[1:-1] + '\0'
        ser.write(bytes([len(xToSend)]))
        ser.write(xToSend.encode())
        startTimer = time.time()
        v = float(ser.readline().decode('utf-8'))
        endTimer = time.time()
        x = np.dot(A, x) + v * B
        y.append(np.dot(C, x).item(0))
        timer.append(endTimer - startTimer)
        u.append(v)
\end{minted}
\caption{Główna pętla przeprowadzająca test HIL}
\label{lst:serial_py}
\end{listing}

\begin{listing}[htb]
\begin{minted}[linenos, breaklines]{python}
_, fig = plt.subplots(2, 1, figsize=(6.4, 9.6))
fig[0].set_title('Wartość wyjścia obiektu w zależności od chwili czasu')
fig[0].plot(y, 'ro'), fig[0].set_ylabel('y'), fig[0].set_xlabel('i')
fig[1].set_title('Wartość sterowania w zależności od chwili czasu')
fig[1].plot(u, 'bo'), fig[1].set_ylabel('u'), fig[1].set_xlabel('i')
plt.savefig(r'{0}\plots\horizons_{1}.png'.format
            (projectPath, systemParameters['horizons']), format='png', bbox_inches='tight')

with open('log.txt', 'a') as logger:
    logger.write(str(systemParameters['A']))
    logger.write('Init time, Mean of timestamps,'
                 ' Stdev of timestamps, Max of timestamps\n')
    logger.write(f"{initTimer:.4} {statistics.mean(timer):.4}"
                 f"{statistics.stdev(timer):.4} {max(timer):.4}\n")
\end{minted}
\caption{Zapisywanie danych i wykresów}
\label{lst:write_py}
\end{listing}

\section{Problemy napotkane podczas realizacji} \label{sec:problems}
HAL - przerwania, ramka UARTa
Model układu - feasibility
